\documentclass[a4paper, 12pt]{article}
\usepackage[utf8]{inputenc}
\usepackage[american]{babel}
\usepackage[margin=1in]{geometry}
\usepackage{mathtools}
\usepackage{fancyhdr}
\setlength{\headheight}{15.2pt}
\pagestyle{fancy}
\lhead[]{Joseph Petitti}
\rhead[]{Homework 1}
\chead[]{Database Systems II}

\begin{document}

\section*{Problem 1}

\subsection*{Q1}

\[ \text{Disk capacity} = 10 \text{ surfaces} \times 8000 \text{ tracks} \times
		208 \text{ sectors} \times 512 \text{ Bytes}  \]

\[ \text{Disk capacity} = 8519680000 \text{ B} = 8320000 \text{ KB} = 8125
		\text{ MB} = \textbf{7.935 GB} \]

\subsection*{Q2}

\[ 8519680000 \text{ B} \div 8192 \text{ Bytes per block} = \textbf{ 1040000
				blocks} \]

\subsection*{Q3}

\begin{table}[h]
	\centering
	\begin{tabular}{l c c c c c c r}
		& Seek time & & Rotational latency & & Transfer time & &  Total \\
		
		\hline

		Minimum & 0 ms & + & 0 ms & + & 0.8 ms & = & \textbf{0.8 ms} \\

		Maximum & 17 ms & + & 11.1 ms & + & 0.8 ms & = & \textbf{28.9 ms}  \\

		Average & 9 ms & + & 5.6 ms & + & 0.8 ms & = & \textbf{15.4 ms} \\

	\end{tabular}
\end{table}

\subsection*{Q4}

\begin{itemize}
	\item $ 8192 \text{ Bytes per block} \div 128 \text{ Bytes per record} =
			\textbf{64 records per block}  $
	\item $ 100000 \text{ records} \div 64 \text{ records per block} =
			\textbf{1563 blocks} $
	\item $ 1563 \text{ blocks} \times 16 \text{ sectors per block} =
			\textbf{25008 sectors} $
\end{itemize}

\subsection*{Q5}

\[ 5.6 \text{ ms (initial half rotation)} + 1.2 \text{ ms (seek time)} + 8
		\text{ ms (transfer time)} = \textbf{14.8 ms} \]

\subsection*{Q6}

\[ 208 \text{ sectors per track} \div 16 \text{ sectors per block} = 13 \text{
				blocks per track} \]

\[ 13 \text{ blocks per track} * 10 \text{ surfaces} = \textbf{130 blocks per
				cylinder} \]

\subsection*{Q7}

The most efficient way to store blocks in a file, in order to speed up the
sequential read of that file, is to start with $B_1$, $B_2$, ... $B_{10}$
aligned under each other on all ten surfaces of the innermost cylinder. This
way, once the disk arms are in position the first ten blocks can all be read in
0.8 ms. Then the next ten blocks should be in the next sixteen sectors per
surface in the direction of rotation on the same cylinder. This should continue
until all 130 blocks on the innermost cylinder are filled. Then, the next ten
blocks should be written on the second-to-innermost cylinder approximately 10\%
of the circumference of the disk away from where $B_{130}$ was written. Once the
disk arms are done reading $B_{130}$, they need to move out to the
second-to-innermost track, which takes about 1.002 ms. In this time, the disk
will have rotated by about 9.027\%, so if $B_{131}$ starts about 10\% of the
disk circumference away from $B_{130}$ the disk arms will arrive at the right
track just in time to start reading it.  Continue in the same spiral pattern out
from the center of the disk until $B_{last}$.

The average time to read this file is, in milliseconds:

\[ 14.6 + \frac{0.8 n}{10} + \left ( \left \lfloor{\frac{n}{130}}\right \rfloor
				\times 1.002 \right ) \]

Where $n$ is the number of blocks in the file. The 14.6 ms accounts for initial
5.6 ms of rotational latency and 9 ms of seek time. The time to read each block
is 0.8 ms, multiplied by $n$ divided by 10 (because ten blocks are read at a
time). The final component is the floor of $ \frac{n}{130} $, which is the
number of cylinders necessary to store the file, multiplied by the 1.002 ms it
takes to seek to the next cylinder while reading.

A file with 100,000 records, with each record being 128 bytes, would have 1,563
blocks. This file's average read time would be:

\[ 14.6 + \frac{(0.8 \times 1563)}{10} + \left ( \left
						\lfloor{\frac{1563}{130}}\right \rfloor \times 1.002
		\right ) = \textbf{151.7 ms} \]

\section*{Problem 2}

\begin{table}[h]
	\centering
	\begin{tabular}{ l c | l c }
		\multicolumn{2}{c}{4-byte} &
		\multicolumn{2}{c}{8-byte} \\

		Field & Index & Field & Index \\
		
		\hline
	
		Header & 0 & Header & 0\\
		ID & 8 & ID & 8 \\
		Name & 12 & Name & 16 \\
		Age & 40 & Age & 48 \\
		DoB & 44 & DoB & 48 \\
		Gender & 56 & Gender & 72 \\
		Address & 60 & Address & 80 \\
		State & 120 & State & 144
	\end{tabular}
\end{table}

\subsection*{Q1}

Each record would be 128 bytes.

\subsection*{Q2}

Each record would be 152 bytes.

\subsection*{Q3}

\subsubsection*{4-Byte Boundaries}

\[ 4096 \text{ B} = 64 \text{ B} + ( 128 \text{ B} \times n ) \]

\[ n = \textbf{31 records} \]

\subsubsection*{8-Byte Boundaries}

\[ 4096 \text{ B} = 64 \text{ B} + ( 152 \text{ B} \times n ) \]

\[ n = \textbf{26 records} \]

\end{document}
